%%%%%%%%%%%%%%%%%%%%%%%%%%%%%%%%%%%%%
% Read the /ReadMeFirst/ReadMeFirst.tex for an introduction. Check out the accompanying book "Better Books with LaTeX" for a discussion of the template and step-by-step instructions. The template was originally created by Clemens Lode, LODE Publishing (www.lode.de), mail@lode.de, 8/17/2018. Feel free to use this template for your book project!
%%%%%%%%%%%%%%%%%%%%%%%%%%%%%%%%%%%%%


% If you have added or removed any entries in the glossary directory, add them here. If a letter is missing, add a new \section*{} with the letter.

\chapter{Glossary}

% Special formatting for glossary 
\setlength{\parindent}{0.7cm}
\renewcommand{\index}[1]{}
\renewenvironment{definition}[2][]{\textbf{#2}\ $\bullet$\ #1}
\footnotesize
\ifxetex
	\titlespacing*{\section}{0pt}{3.5ex plus 1ex minus .2ex}{2.3ex plus .2ex}
\fi

\vspace{20pt}


\section*{E}
\begin{multicols}{2}
\begin{definition}{Entity}\index{entity|textbf} An \emph{entity} is a ``thing'' with properties (an identity). For example, a plant produces oxygen, a stone has a hard surface, etc.).\end{definition}
\end{multicols}

\section*{I}
\begin{multicols}{2}
\begin{definition}{Identity}\index{identity|textbf} An \emph{identity} is the sum total of all properties\index{entity!property} of an entity (e.g., weight: 160 pounds, length: 6 feet, has a consciousness, etc.).\end{definition}

\end{multicols}

\section*{L}
\begin{multicols}{2}
\begin{definition}{LaTeX}\index{latex|textbf} \LaTeX{} is a document preparation system.\end{definition}
\end{multicols}

\section*{P}
\begin{multicols}{2}
\begin{definition}{Property}\index{entity!property|textbf} A \emph{property} refers to the manner in which an entity (or a process) affects other entities (or other processes) in a certain situation (e.g., mass, position, length, name, velocity, etc.).\end{definition}
\end{multicols}